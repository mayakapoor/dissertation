\section{Previous Work}

Applications for regular expression signature matching to packet payloads originate in content inspection for malware detection. Early work in automatic signature generation for polymorphic worms such as Autograph \cite{KimHyangAh} provided a foundation for signature-based analysis. This expanded into detecting personally identifiable information as well as policy violations such as copyright infringement, inappropriate or sensitive information on enterprise networks, and censorship. Additionally, Tang et al \cite{TANG2009827} found multiple sequence alignment to be effective in rewarding consecutive substring extractions and tolerating noise in traffic. SigBox~\cite{sigbox} introduced the technique of generating substring tokens from packet payloads and applying Apriori data mining to find the most frequent ones. The related CSP Algorithm \cite{Sija} and works by Wang et al \cite{WANG2012992}, Szabo et al \cite{Szabo}, LASER \cite{LASER}, and AutoSig \cite{AutoSig} all include feature extraction and sequence alignment. Wang et al describe their system as a four-stage process; we generalize this concept to all these solutions and our own in order to compare various approaches.

In the first step, data is pre-processed in order to prepare it for use at the next stage. This typically involves extraction of certain fields or N number of bytes from the packet payload. For all these systems, sessionization of packet flows and sometimes defragmentation and reassembly based on TCP/IP header values is required \cite{sigbox, WANG2012992, Sija, Szabo, AutoSig, LASER}. Once the data is collected and prepared, the second stage finds common substrings across packet data and/or protocol flows. The systems examined which perform feature extraction use either a subtree approach \cite{WANG2012992}, a longest common substring algorithm \cite{AutoSig, LASER}, motifs \cite{Szabo}, or sequential pattern mining \cite{sigbox, Sija}. Third, a method of alignment is used to align data based on commonalities between packets. Wang et al, Szabo et al, Vinoth et al \cite{VinothGeorge2013EfficientRE} and LASER use bioinformatics approaches to perform these alignments. A substring tree can also be used for this purpose \cite{AutoSig}. Some of these alignment strategies are additionally informed by a scoring matrix influenced by the tokens derived in the previous step \cite{WANG2012992, Szabo, LASER}. Once the sequences are aligned in an optimal manner, in the fourth and final stage the systems convert their results into regular expressions.
