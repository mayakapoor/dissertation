\section{Previous Work}

Jaccard similarity measures the intersection of two sets over their union. This value is often used in recommender systems, and can also be used to detect plagiarism or make predictions. In order to take the Jaccard similarity, data must be split up into sets, which are made up of individual elements like the tokens or shingles of strings as done in SigBox~\cite{sigbox} and RExACtor~\cite{rexactor} for regular expression signature generation.

Some limited work has been done in application fingerprinting through locality-sensitive hashing. Tang et al proposed HSLF~\cite{hslf}, an HTTP header sequence-based LSH fingerprint generator for classifying applications in HTTP traffic. Their results show the ability of their SimHash-based method to distinguish accurately among data such as VMWare, Firefox, and WeChat. This work is limited as it only applies to cleartext HTTP traffic. Furthermore, the scope of this traffic only makes up a fraction of real-world data.


Jiang and Gokhale~\cite{fpga} use locality-sensitive hashing on packet-level features to accurately classify network traffic statelessly. From a model perspective, their use of K-Nearest Neighbors clustering scales sub-optimally. Their work also focused solely on multimedia applications versus legacy web-browsing; there is no expansion into traffic type, application, protocol, or other needed classification in current DPI. \textsc{Alpine} and \textsc{Palm} expand these works into a much more complex and diverse set of experiments and applications. 
