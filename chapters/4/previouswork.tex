\section{Previous Work}

The Jaccard similarity of sets is commonly used to recommend products or next steps in a workflow, or detect plagiarism, or make predictions. These sets must be made up of individual items like the tokens or shingles of strings as done in SigBox~\cite{Shim2017SigBoxAS} and RExACtor~\cite{rexactor}. Both these works use the tokens as part of regular expression generation.

A limited amount of work has been done in applying locality-sensitive hashing in order to generate application fingerprints. Tang et al proposed HSLF~\cite{hslf}, an HTTP header sequence-based LSH fingerprint generator for classifying applications in HTTP traffic. While this work is limited to only cleartext HTTP traffic, results show the ability of their SimHash-based method to accurately distinguish between data such as Firefox, VMWare, and WeChat. Naturally, this work has limitations due to encryption and the scope of traffic which only makes up a fraction of real-world data.

Jiang and Gokhale~\cite{fpga} show the ability of locality-sensitive hashing to accurately classify network traffic in their research using packet-level features exclusively in order to achieve stateless packet inspection. Their research focused solely on multimedia applications versus legacy web-browsing; they did not expand into encapsulation architectures and more traffic classes as well as application-level identification. They also use K-Nearest Neighbors clustering for classification, which is known to scale sub-optimally. \textsc{Alpine} and \textsc{Palm} expand these works into a much more complex and diverse set of experiments and applications.
