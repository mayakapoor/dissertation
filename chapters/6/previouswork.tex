\section{Previous Work}

\textsc{DATE} is a novel contribution to the field of packet processing as there is little published, previous work in generating 3D point cloud representations of packets. Raw packet data from LiDAR (light detection and radiation) systems have been shaped losslessly into 2D matrices and point clouds~\cite{Tu2019point}. This work introduces the problem of spatial correlation in packet data, namely that packets in their raw state are not usually uniform or in a state which may be processed using the spatial or geometric strategies employed by image and computer vision algorithms. Thus, data compression or pre-processing must be performed to create the necessary uniformity.

Costeux et al~\cite{costeux2006detection} worked on the fast detection of Skype and other RTP-based telephony traffic. They specify several fields from the RTP and encapsulation header which can be used to filter out non-RTP traffic. We employ this technique in the middlebox as an initial filtering step in our end-to-end data processing pipeline.

Kmet et al~\cite{kmetfast} build on Costeux by incorporating additional header features for per-packet selection, and adding a flow-based solution to reduce the over-selection problem. They were able to minimize mis-classification to nearly zero by buffering up to 10 packets of the RTP stream. The synchronization source number from the header is used along with timestamps to check for proper increment and stream correlation. We focus on per-packet identification only.

Some early research~\cite{patwari2005manifold} describes expanding NetFlow data generated by routers into 2D manifolds. We reference this as an intial introduction of spatial expansion of traffic data in order to perform dimensionality reduction, which is a key theoretical linkage to our work.
