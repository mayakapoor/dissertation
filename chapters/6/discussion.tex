\section{Discussion and Limitations}

Of the models tested so far in this work, \textsc{Date} is the most computationally complex with the lowest throughput. In future work, we propose implementing multi-threading and multi-core processing as potential optimizations. We observed that point cloud generation was a pain point for the system in terms of cycles, and propose offloading such repetitive calculations to a specialized hardware such as FPGA~\cite{Song2005EfficientPC} when available in the deployed system. Because DATE is able to normalize data and map it into a three-dimensional cluster space, it would be well-suited to heterogeneous data such as sensor data combined with packet data from an internet of things device. This would create a multi-embedding capable of correlating these data inputs for machine learning tasks. Simpler problems may be solved with more naïve approaches such as regular expression matching on plaintext SIP traffic, or using the locality-sensitive hashing strategy toward device fingerprinting.

We propose both \textsc{Maple} and \textsc{Date} as potential methods for generating hidden, latent-space representations of traffic as their capabilities extend to different problem areas. \textsc{Maple} works well on input data such as a payload which may be distinct one data sample from another, but can be divided and matricized into units for comparison (i.e. turned into a grayscale image of uniform dimension). On the other hand, \textsc{Date} has the potential to expand to include other non-network features and represent more heterogenous data. For example, input sensor data from IoT devices or light detection and ranging (LiDAR) equipment have used DBSCAN for data processing and normalization~\cite{wanglidar2019}; it could be combined with cloud/network data inputs as an additional embedding model for classification problems. While for the RTP problem \textsc{Maple} provides high accuracy with more efficiency than \textsc{Date}, there is a trade-off within the embedding space as \textsc{Date} may be able to represent data of higher complexity in other problems for future work.
