\section{Future Work}
The scope of this dissertation includes the research and methods leading to the initial release of the Forager toolkit. This work focuses on per packet inspection and data transformation techniques which are viable for large-scale deployments. There are a number of flow-based techniques referenced in this work which do have practical applications. Netflow data is often used in recurrent neural network or long short-term memory network analysis and can be extracted often with ease from routers, switches, or other network devices. A future expansion of this toolkit would be the ability to process netflow or other such stream-based data or cross-packet features when available.

There are also many deep learning and data mining techniques which we have not yet covered but have been explored in other works. Principle component analysis (PCA)~\cite{Yan2014Principal, Nilesh2021machine} and eigenvectors~\cite{Qing2021Encrypted} have been used in per packet classification with success. These techniques are notably efficient and therefore would be excellent candidates for additional features in a future release for processing at large scale.

While we found that adjusting for TF-IDF score in \textsc{Palm} did not improve classification accuracy in the experiments we ran, there are applications for other natural language processing and word embedding techniques in the literature on traffic classification. Word embeddings can be used to classify real versus fake domain names~\cite{Vinayakumar2020visualized}, phishing emails~\cite{rao2022application,  yuan2018url2vec, tajaddodianfar2020texception} and other kinds of phishing attacks~\cite{somesha2022classification}, and HTTP traffic~\cite{gniewkowski2021http2vec}. In the cybersecurity domain, Word2Vec has been used to identify cross-site scripting~\cite{fang2018deepxss, guichang2019cnnpayl} and SQL injection attacks~\cite{yu2019detecting} across network packets. Textual analysis is rendered ineffective by encryption and compression in many cases, but for plaintext analysis or malware detection could be a future approach.

The data itself we chose to work with in these experiments until \textsc{Forager}'s initial release has largely been sourced from traditional network devices. Mobile traffic and data gathered from IoT or medical/wearable devices could present unique challenges and opportunities for classification problems. In future versions of this toolkit, we could leverage the dimensionality reduction offered by our data engineering and mining techniques in order to combine feature sets and further increase multimodality. For example, we could embed both the processing power and CPU utilization of an IoT device input along with network traffic embeddings to further inform a classification of IoT devices. There is also much opportunity for experimentation with classifying streams of traffic in multiple ways. As mentioned in the introduction of this work, traffic classification covers a broad scope of problems; importantly, these classes are not distinct from one another and traffic may belong to multiple classes and it may be necessary to classify something multiple times. For example, we may want to determine the user who generated a traffic stream (Bob), what protocols are in the traffic (SIP, RTP), what application Bob used to generate the traffic (Zoom), and what device of Bob's he used to make the call (MacBook Air). Determining methods to perform multiple classifications like this without re-processing the data each time would be a useful additional contribution.

\section{Summarization}
This work proposes a new system of data mining and deep learning for network traffic classification at scale. The \textsc{Forager} system includes models which contribute both novel application of data transformation techniques which unveil hidden representations as well as machine learning models applied in real systems. We propose a method for extracting data from PCAPs and converting them to tabular format, then transforming them appropriately according to what we have learned from previous experiments. For example, plaintext, strongly signatured data which we are aiming to identify by protocol (SIP, HTTP, POP) may be aptly suited for regular expression generation algorithms. Alternatively, weakly signatured data may be run with the \textsc{Maple} model for payload analysis and \textsc{Alpine} model for header features for identification. Similarly for profiling, encrypted data may also be well-suited for the \textsc{Alpine} and \textsc{Maple} model combination. Our ensemble voting system allows for this multimodal analysis and can finally be saved and returned through a tagging system implemented in the original data. Thus, our toolkit will be a useful contribution to network analysts and cybersecurity and surveillance specialists for uniquely identifying traffic on the internet today.
