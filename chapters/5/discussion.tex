\section{Discussion and Limitations}
\textsc{Maple} works well on input data such as a payload which may be distinct from one data sample from another, but can be divided and matricized into units for comparison (i.e. turned into a grayscale image of uniform dimension). For deep packet inspection, this model is able to create latent representations of payload data which uniquely identify traffic of different types at a higher dimension than is considered by current signature-matching or filter-based solutions used in industry. Our implementation for this initial deployment solves the RTP detection problem with high accuracy using a minimal framework ideal for line-rate. The system could be further optimized by running these models in parallel; as the solution relies on per-packet analysis, throughput could be increased across models. One limitation of the proposed encoding is that individual packets must be large enough to create a significant image. Thus, this model would not work well on short packets or packets with no payload data. In this case, we propose using a combination of header features and payload when available would be a wiser strategy.
