\section{Introduction}

In chapters 3 and 4, we discuss the difficulty of generating token-based or regular expression-based signatures for weakly signatured, variable data. This type of data floods modern Internet traffic today; examples would be Voice-over-IP (VoIP), streaming services, downloads and file transfers, and peer-to-peer sharing. Furthermore, encrypted traffic payloads or compressed data make generating signatures from payload text ineffective. The tokenization strategy of \textsc{Palm} is defeated by these techniques, and as \textsc{Alpine} focuses only on header features there is still the problem of an unanalyzed payload with potentially useful, untapped information.

In 2021, it was projected that 3 billion people use VoIP technology as a regular part of their daily lives. In 2021-2022, Zoom reported over 300 million users~\cite{teamstage}, Webex reported over 600 million, and Microsoft Teams recorded 275 million participants~\cite{businessofapps}. The widespread adaptation of VoIP technology has highlighted several cybersecurity vulnerabilities which make this type of traffic even more important to network intrusion detection systems. It is simple to spoof IP addresses or Session Initiation Protocol (SIP) universal resource identifiers with VoIP technology and produce robocalls used in phishing/spam or in denial of service attacks~\cite{edwards2020robocalling}. VoIP packet data is vulnerable to being the transport for backdoors, worms, trojans, and viruses which can be embedded~\cite{Wu2021SteganographyAS, nagaraja2019voiploc}. Cybersecurity specialists designing intrusion detection systems need to be able to intercept and process VoIP streams in order to spot these kinds of intrusions~\cite{choti2021prediction}. VoIP call analysis and reconstruction can also be used in the forensic environment to provide critical evidence of criminal activity. Policy or content rights violations or copyright infringement may also be detected through the analysis of the reconstructed calls~\cite{kmetfast, Sha2016VoIPFA}. This process and search capability is also useful to intelligence operations to find mission critical data from the enormous amount of traffic flowing in mission networks~\cite{kao2020forensic}.

A majority of mainstream VoIP protocols use the real-time transport protocol (RTP) as their transport layer for encapsulation. The application layer protocol, for example the Session Initiation Protocol (SIP), Media Gateway Control Protocol (MGCP), or H.248, will establish the connection between endpoints and carry important session information such as codec encoding and metadata for the call. The application then relies on RTP to manage the dataflow between the connected systems. In the \textit{complex, real network environment}, the implementation of SIP and RTP data flow is often sent de-coupled across physical signals and ports which presents a challenge for cybersecurity specialists using middlebox technologies for packet inspection and call reconstruction. Because the RTP data is encoded, it is necessary to know the signaling information in order to retrieve the codec information for proper decoding. Furthermore, the metadata associated with signaling protocols such as the SIP URI identifier provides necessary enrichment and context for the actual call. Lastly, RTP data may arrive at the middlebox before the signaling information which contains the port numbers associated with the RTP stream for that particular call; thus, the middlebox must retrospectively identify data packets (RTP) which belong to particular signaling information headers (SIP, for example).

Detecting the RTP stream itself from other types of traffic can be a difficult problem because the protocol's signature is weak and can be further obfuscated by encryption (SRTP). The problem of accurately classifying network traffic has expanded beyond the scope of capability of text-based solutions. Instead, we propose using higher dimensionality of network traffic in order to advance RTP detection capability. \textsc{Maple} is a \textbf{MA}trix-based \textbf{P}ay\textbf{L}oad \textbf{E}ncoder which transforms packets into grayscale images to unveil hidden representation which may be used as input into a convolutional neural network model. Using this approach, we are able to expand detection capability to weakly signatured data, encrypted versus compressed traffic, and better traffic type and application profiling.
