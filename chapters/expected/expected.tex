As a contribution for this work, we provide the \textsc{Forager} toolkit as a usable software package and system for individual use. This software is a command line interface (CLI) application installed through PyPi and user documentation provided through ReadTheDocs. The following components are included in the initial 1.0.0 release:

\medskip

\textbf{TaPCAP: } This option parses input PCAP and PCAPNG files, extracting header features and/or payloads into CSV format. This module can be installed separately~\footnote{https://pypi.org/project/tapcap/} or run from \textsc{Forager}.

\textbf{\textsc{RExACtor}: } The token option finds frequent tokens according to a provided threshold from a CSV output from \textsc{Tapcap}. The regular expression mode performs genetic sequencing alignment of payloads from a CSV output from \textsc{Tapcap}. \textsc{Rexactor} can be installed as a separate component~\footnote{https://pypi.org/project/rexactor/} or run from \textsc{Forager}.

\textbf{\textsc{Alpine}: } In training mode, this model accepts as input the CSV obtained from \textsc{Tapcap}, generates locality sensitive hashes from the header data, and finalizes and serializes these to an LSHForest~\cite{lshforest}. In testing mode, the trained LSHForest is reloaded from the internal cache. Locality sensitive hashes of the input data are generated and index lookup performed. The returned result is applied to the ensemble voting system if other models are configured.

\textbf{\textsc{Palm}: } In training mode, this model accepts as input the CSV obtained from \textsc{Tapcap}, generates locality sensitive hashes from the payload data, and finalizes and serializes these to an LSHForest~\cite{lshforest}. In testing mode, the trained LSHForest is reloaded from the internal cache. Locality sensitive hashes of the input data is generated and index lookup performed. The returned result is applied to the ensemble voting system if other models are configured.

\textbf{\textsc{Maple}: } In training mode, this model accepts as input the CSV obtained from \textsc{Tapcap}, generates grayscale images from the payload data, and saves the generated neural network model and trained weights to JSON and H5 files, respectively. In testing mode, the trained neural network is reloaded from the internal cache. Grayscale images of the input data are generated and used as input to the CNN classifier. The returned result is then applied to the ensemble voting system if other models are configured.

\textbf{\textsc{Date}: } In training mode, this model accepts as input the CSV obtained from \textsc{Tapcap}, generates point clouds from the payload data and performs DBSCAN analysis. It then saves the generated neural network model and trained weights to JSON and H5 files, respectively. In testing mode, the trained neural network and weights are reloaded from the internal cache. Point clouds are generated from input data and DBSCAN analysis performed, and this is used as input to the multi-layer perceptron classifier. The returned result is applied to the ensemble voting system if other models are configured.

\textbf{Multi-modality: } \textsc{Forager} supports training and using one or more of the \textsc{Alpine}, \textsc{Palm}, \textsc{Maple}, and \textsc{Date} models together so that the packet may be analyzed from multiple representations. We use votes in a combined ensemble classifier and return the class with the most votes as the final classification.
