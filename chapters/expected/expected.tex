\textsc{Rexactor} furthers the state-of-the-art in regular expression generation by:

\begin{itemize}
\item adding a component tool for automatically extracting frequent and certain tokens from packet payloads,
\item applying genetic sequencing for substring alignment combined with frequent tokens,
\item encoding substrings and tokens in an original algorithm for more enriched, expressive regular expressions,
\item adding an additional tool for regular expression scanning using state-of-the-art automata software. \end{itemize}

\textsc{Palm} and \textsc{Alpine} provide the following contributions:

\begin{itemize}
\item A process for generating multiple locality-sensitive hash embeddings from packets which is highly accurate for identification of many classes, including protocol type, traffic type, application, and more,
\item A generalizable framework which applies to many network traffic classification problems, and whose model can be quickly trained and adapted to suit new problems,
\item An alternative to regular expression scanning for DPI which scales sublinearly and requires a linear amount of storage space,
\item A diverse and unique application of the traffic classification problem with experiments classifying multiple classes of protocols across many traffic types,
\item and public datasets and source code for experimental reproducibility.
\end{itemize}

\textsc{Maple} and \textsc{Date} additionally contribute:

\begin{itemize}
\item An algorithm for encoding packet payloads to image-based embeddings for latent representation,
\item an empirical evaluation and comparison of CNN models on RTP data,
\item an algorithm for generating three-dimensional point clouds from packet data, creating spatial latent representations as a novel method of packet analysis,
\item a novel application of density-based cluster analysis on packet payloads.
\end{itemize}

As a contribution for this work, we propose realizing the \textsc{Forager} toolkit as a usable software package and system for individual use. We project an initial release of this software by June 2023. System progress and planned development can be viewed in the Gantt chart in Figure~\ref{fig:gantt}. The final deliverable of this system will be a command line interface (CLI) application installed through PyPi. We will provide the following components in the initial release.

\textbf{TaPcap: } This option will parse input PCAP files, extracting header features and/or payloads into CSV or text file format.

\textbf{\textsc{TRex}: } This option will find frequent tokens according to a provided threshold from a CSV output from \textsc{TaPcap}.

\textbf{\textsc{GRex}: } This option will perform genetic sequencing alignment of payloads from a CSV output from \textsc{TaPcap}.

\textbf{\textsc{Alpine}: } In training mode, this model will accept as input the CSV obtained from \textsc{TaPcap}, generate locality sensitive hashes from the header data, and finalize and serialize these to an LSHForest~\cite{lshforest}. In testing mode, the trained LSHForest will be reloaded. Locality sensitive hashes of the input data will be generated and index lookup performed. The returned result will be applied to the ensemble voting system if other models are configured.

\textbf{\textsc{Palm}: } In training mode, this model will accept as input the CSV obtained from \textsc{TaPcap}, generate locality sensitive hashes from the payload data, and finalize and serialize these to an LSHForest~\cite{lshforest}. In testing mode, the trained LSHForest will be reloaded. Locality sensitive hashes of the input data will be generated and index lookup performed. The returned result will be applied to the ensemble voting system if other models are configured.

\textbf{\textsc{Maple}: } In training mode, this model will accept as input the CSV obtained from \textsc{TaPcap}, generate grayscale images from the payload data, and save the generated neural network model and trained weights to JSON and H5 files, respectively. In testing mode, the trained neural network will be reloaded. Grayscale images of the input data will be generated and used as input to the CNN classifier. The returned result will be applied to the ensemble voting system if other models are configured.

\textbf{\textsc{Date}: } In training mode, this model will accept as input the CSV obtained from \textsc{TaPcap}, generate point clouds from the payload data and perform DBSCAN analysis, and save the generated neural network model and trained weights to JSON and H5 files, respectively. In testing mode, the trained neural network will be reloaded. Point clouds will be generated from input data and DBSCAN analysis performed, and this will be used as input to the multi-layer perceptron classifier. The returned result will be applied to the ensemble voting system if other models are configured.

\textbf{Multi-modality: } \textsc{Forager} will support using one or more of the \textsc{Alpine}, \textsc{Palm}, \textsc{Maple}, and \textsc{Date} models together so that the packet may be analyzed from multiple representations. We will use votes in a combined ensemble classifier and return the index with the most votes as the classification.

\textbf{Tagging: } \textsc{Forager} will write the classification to a column in the \textsc{TaPcap} file.

\begin{figure*}
\includegraphics[width=\textwidth,height=\textheight]{chapters/expected/img/gantt.png}
\caption{Gantt chart of \textsc{Forager} 1.0.}
\label{fig:gantt}
\end{figure*}
