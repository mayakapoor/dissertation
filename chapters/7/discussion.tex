\section{Discussion and Limitations}

Using \textsc{Forager}, we are able to provide new insight into a previously under-analyzed gateway into secure networks - port 443. These techniques enable profiling of encrypted traffic streams while preserving user privacy of the content. We can also use \textsc{Forager} to identify plaintext or compressed traffic from the encrypted streams which may then be forwarded to decompression and DPI tools. Finally, we show the ability of the system to identify not only particular protocols in the encrypted stream, but also distinguish between benign and malicious instances of their utilization. This would be useful, for example, applied in network intrusion detection (NIDS) and prevention systems. For surveillance missions, we show \textsc{Forager} keeps up with the ever-changing, fast-paced real and complex network environment and allows analysts to stay ahead of mission communications and critical data. In seeing through encryption and tunneling, \textsc{Forager} proves to be a versatile solution for real network problems.

For plaintext analysis and keyword searching, using regular expression scanning can still be an effective technique. Automatic generation techniques like RExACtor may also reduce manual overhead required for signature creation and maintenance, and provide useful insight to commonalities across traffic types and protocols. Additional regular expression scanning and generation tools would be a useful addition to the \textsc{Forager} toolkit.
